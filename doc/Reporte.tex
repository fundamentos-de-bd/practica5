\documentclass{article}
\usepackage[utf8]{inputenc}
\usepackage[spanish]{babel}

% Formato de página
\usepackage[letterpaper, margin = 1.5cm]{geometry}

% Más opciones para enumerar
\usepackage{enumitem}

% Manejo de ecuaciones
\usepackage{amsmath}

% Manejo de imágenes
\usepackage{float}
\usepackage{graphicx}
\usepackage{wrapfig}
\graphicspath{{img/}}

\begin{document}
    \title{
        Fundamentos de bases de datos \\
        Práctica 4 \\
        Modelo Relacional
    }
    \author{
        Díaz Gómez Silvia \\
        Eugenio Aceves Narciso Isaac \\
        Quiroz Castañeda Edgar
    }
    \date {
        22 de marzo del 2019    
    }
    \maketitle
    \section{Modificaciones al esquema}
    \begin{figure}[H]
    	\centering
        \includegraphics[scale=0.22]{img/practica04.jpeg}
        \caption{Esquema anterior}
    \end{figure}

    \begin{figure}[H]
    	\centering
	  \includegraphics[scale=0.22]{img/practica05.jpeg}
	  \caption{Esquema modificado}
    \end{figure}
    Modificaciones a entidades
    \begin{itemize}
        \item A Empleado se le añaden los atributos de puesto y fecha reg, la
        fecha en la que empezó a trabajar el empleado.
        \item Se cambia el discriminante de Departamento de Localización a Id dep.
        \item A Sucursal se le agrega el atributo de fecha fund, la fecha de 
        establecimiento de esa sucursal.
        \item A Sucursal se le agrean los datos de su dirección, es particular su
        estado.
        \item A Persona se le agrega la fecha nac, la fecha de nacimiento.
        \item A Producto se le agrega el atributo de descr, la descripción. Y 
        se remplaza la llave compuesta por una llave sintética.
    \end{itemize}

    Modificaciones a relaciones
    \begin{itemize}
        \item Se introduce la tabla de ``Trabajar'' pues es posbile que un
        empleado tenga más de un trabajo.
    \end{itemize}
    
    \section{Álgebra relacional}
    Utilizando el diagrama relacional que haya creado deberán escribir las
    siguientes consultas utilizando los elementos del álgebra relacional.
    \begin{enumerate}
        \item {
            Conocer los datos de las sucursales que tengas más de 15 años
            \begin{align*}
                r &\leftarrow (_{id\_suc}G_{((fecha\_actual-fecha\_reg)/365.25) > 15}(Sucursal)) \\
                %r &\leftarrow \rho_{edad(tiempoDesde(fecha\_reg))}(r) \\
                %r &\leftarrow \sigma_{edad > 15}(r) \\
                %r &\leftarrow r \bowtie Sucursal
            \end{align*}
        }
        \item {
            Conocer el puesto, nombre, edad y la fecha en la que iniciaron a
            trabajar de todos los empleados.
            \begin{align*}
                r &\leftarrow Empleado \bowtie Personas \\
                r &\leftarrow (_{CURP}G_{((fecha\_actual-fecha\_reg)/365.25)}(r))\\
                r &\leftarrow \rho_{edad((fecha\_actual-fecha\_reg)/365.25)}(r) \\
                r &\leftarrow \pi_{nombre, puesto, edad, fech\_reg}(r)
            \end{align*}
        }
        \item {
            Conocer el puesto y edad de todos los empleados que trabajan en más 
            de una sucursal.
            \begin{align*}
                r &\leftarrow (_{id\_empleado}G_{count(id\_empleado)}(Trabajar)) \\
                r &\leftarrow \rho_{num\_trabajos(count(id\_empleado)))}(r)\\
                r &\leftarrow \sigma_{num\_trabajos>1}(r)\\
                r &\leftarrow r \bowtie Empleados \bowtie Persona \\
                r &\leftarrow (_{CURP}G_{((fecha\_actual-fecha\_reg)/365.25)}(r))\\
                r &\leftarrow \rho_{edad((fecha\_actual-fecha\_reg)/365.25)}(r) \\
                r &\leftarrow \pi_{puesto, edad}(r)
            \end{align*}
        }
        \item {
            Conocer los productos que se venden de cada sucursal, para esto se 
            debe regresar el identificador de la sucursal, seguido del
            identificador del producto y la descripción de este.
            \begin{align*}
                r &\leftarrow \pi_{id\_dep, id\_suc}(Departamento \bowtie Sucursal)\\
                r &\leftarrow \pi_{id\_suc, id\_prod, descripcion}(r \bowtie Producto)
            \end{align*}
        }
        \item {
            Conocer los departamentos que tienen cada una de las sucursales.
            \begin{align*}
                r \leftarrow \sigma(Departamento)
            \end{align*}
        }
        \item {
            Conocer cuales son los departamentos que tienene un común todas las 
            sucursales.
            \begin{align*}
                r &\leftarrow (_{tipo}G_{count(Tipo)}(Departamento)) \\
                r &\leftarrow \rho_{num\_tip(count(Tipo))}(r) \\
                g &\leftarrow count_{(id\_suc)}(Sucursal) \\
                r &\leftarrow \sigma_{num\_tip = g}(r)
            \end{align*}
        }
        \item {
            Conocer el cliente más antiguo (el primero en ser registrado, según
            la fecha de registro) en el programa de tarjeta digital de cada una
            de las sucursales registradas.
            \begin{align*}
            r &\leftarrow _{id\_suc,id\_cliente}G_{((fecha\_actual-fecha\_reg)/365.25)}(Clientes \bowtie Sucursal)\\
            r &\leftarrow \rho_{antig((fecha\_actual-fecha\_reg)/365.25)}(r)\\
            r &\leftarrow _{id\_suc,id\_cliente}G_{Max(antig)} (r) \\           
            \end{align*}
        }
        \item {
            Conocer cuáles son los productos que tienen en común cada uno de los
            departamentos de las diferentes sucursales.      
            \begin{align*}
                r &\leftarrow \pi_{id\_suc}(Sucursal) \bowtie \pi_{id\dep}
                (Departamento) \bowtie \pi_{id\_prod}(Producto)
            \end{align*}
        }
        \item {
            Conocer cuales son TODOS los productos que se tienen en cada uno de
            los departamentos de las diferentes sucursales.
            \begin{align*}
                r \leftarrow Producto
            \end{align*}
        }
        \item {
            Conocer cuál es la sucursal con mayor número de productos
            registrados en sus diferentes departamentos.
            \begin{align*}
            r &\leftarrow _{id\_suc}G_{count(id\_prod)}(Producto \bowtie Departamento)\\
            r &\leftarrow \rho_{cant\_prod(count(id\_prod))}(r)\\
            s &\leftarrow Max_{cant\_prod}(r)\\
            r &\leftarrow \sigma_{cant\_prod=s}(r)\\
            \end{align*}
        }
        \item {
            Eliminar a los empleados que tengan más de 3 trabajos en diferentes 
            sucursales.
            \begin{align*}
                r &\leftarrow (_{id\_empleado}G_{count(id\_empleado)}
                (\pi_{id\_empleado}(Trabajar))) \\
                r &\leftarrow \rho_{num\_trabajos(count(id\_empleado)))}(r)\\
                r &\leftarrow \sigma_{num\_trabajos>3}(r)\\
                r &\leftarrow \pi_{id\_empleado}(r) \bowtie Empleado \\
                Empleado &\leftarrow Empleado - r
            \end{align*}
        }
        \item {
            Eliminar a las sucursales que tengan menos de 1 departamento
            registrado.
            \begin{align*}
                r &\leftarrow \pi_{id\_suc}(Departamento)\\
                r &\leftarrow r \bowtie Sucursal \\
                Sucursal &\leftarrow r
            \end{align*}
        }
        \item {
            Eliminar a los clientes que no hayan utilizado su tarjeta en los
            últimos tres meses.
            \begin{align*}
                r &\leftarrow \pi_{id\_venta, fecha}(Venta) \bowtie TipoDePago \\
                r &\leftarrow \pi_{num\_trans, fecha}(r) \bowtie Metodo \\
                r &\leftarrow \pi_{num\_tarjeta, fecha}(r) \bowtie Tarjeta \\
                r &\leftarrow \pi_{id\_cliente, fecha}(r) \\
                r &\leftarrow (_{id_cliente}G_{max(fecha)}(r)) \\
                r &\leftarrow \rho_{ult\_compra(max(fecha))(r)} \\
                r &\leftarrow \sigma_{ult\_compra/30 > 3}(r) \\
                r &\leftarrow \pi_{id\_cliente}(r) \bowtie Cliente \\
                Cliente &\leftarrow Cliente - r
            \end{align*}
        }
        \item {
            Insertar una nueva sucursal en el estado de México.
            \begin{align*}
                maxId &\leftarrow Max_{id\_suc}(Sucursal)\\
                Sucursal &\leftarrow Sucursal 
                \cup \{(id\_suc = maxId+1, estado='Estado de Mexico', 
                fecha\_fund = hoy)\}
            \end{align*}
        }
        \item {
            Insertar la información de 3 departamentos a la sucursal que fue
            insertada anteriormente.
            \begin{align*}
                id\_s &\leftarrow Max_{id\_suc}(Sucursal)\\
                maxId &\leftarrow Max_{id\_dep}(Departamento)\\
                nSuc &\leftarrow \{(id\_dep = maxId+1, id\_suc=id\_s),
                (id\_dep = maxId+2, id\_suc=id\_s),\\
                &(id\_dep = maxId+3, id\_suc=id\_s)\}\\
                Departamento &\leftarrow Departamento \cup nSuc
            \end{align*}
        }
        \item {
            Actualizar el número de departamentos de la sucursal con menos
            número de éstos, para que ahora tenga la misma cantidad de departamentos
            que la sucursal con mayor números de departamentos.
        }
    \end{enumerate}

\end{document}