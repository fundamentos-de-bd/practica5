\documentclass{article}

% Formato de página
\usepackage[letterpaper, margin = 1.5cm]{geometry}

% Más opciones para enumerar
\usepackage{enumitem}

% Manejo de imágenes
\usepackage{graphicx}
\usepackage{wrapfig}
\graphicspath{{img/}}

\begin{document}
    \title{
        Fundamentos de bases de datos \\
        Práctica 4 \\
        Modelo Relacional
    }
    \author{
        Díaz Gómez Silvia \\
        Eugenio Aceves Narciso Isaac \\
        Quiroz Castañeda Edgar
    }
    \date {
        22 de marzo del 2019    
    }
    \maketitle

    \section{Álgebra relacional}
    Utilizando el diagrama relacional que haya creado deberán escribir las
    siguientes consultas utilizando los eementos del álgebra relacional.
    \begin{enumerate}
        \item {
            Conocer los datos de las sucursales que tengas más de 15 años
        }
        \item {
            Conocer el puesto, nombre, edad y la fecha en la que inició a
            trabajar de todos los empleados.
        }
        \item {
            Conocer el puesto y edad de todos los empleados que trabajan en más 
            de una sucursal.
        }
        \item {
            COnocer lso productos que se venden de cada sucursal, para esto se 
            debe regresar el identificador de la sucursal, seguido del
            identificador del producto y la descripción de este.
        }
        \item {
            Conocer los departamentos que tienen cada una de las sucursales.
        }
        \item {
            Conocer cuales son los departamentos que tienene un común todas las 
            sucursales.
        }
        \item {
            Conocer el cliente más antiguo (el primero en ser registrado, según
            la fecha de registro) en el programa de tarjeta digital de cada una
            de las sucursales registradas.
        }
        \item {
            Conocer cuáles son los productos que tienen en común cada uno de los
            departamentos de las diferentes sucursales.
        }
        \item {
            Conocer cuales son TODOS los productos que se tienen en cada uno de
            los departamentos de las diferentes sucursales.
        }
        \item {
            Conocer cuál es la sucursal con mayor número de productos
            registrados en sus diferentes departamentos.
        }
        \item {
            Eliminar a los empleados que tengan más de 3 trabajos en diferentes 
            sucursales.
        }
        \item {
            Eliminar a las sucursales que tengan menos de 1 departamento
            registrado.
        }
        \item {
            Eliminar a los clientes que no hayan utilziado su tarjeta en los
            últimos tres meses.
        }
        \item {
            Insertar una nieva sucursal en el estado de México.
        }
        \item {
            Insertar la información de 3 departamentos a la sucursal que fue
            insertada anteriormente.
        }
        \item {
            Actualizar el número de departamentos de la sucursal con menos
            número de éstos, para que ahora tenga la misma cantidad de departamentos
            que la sucursal con mayor números de departamentos-
        }
    \end{enumerate}

\end{document}